\documentclass[12pt]{article}

\usepackage[english]{babel}
\usepackage[utf8x]{inputenc}
\usepackage{amsmath}
\usepackage{tikz}
\usetikzlibrary{arrows,automata}

\title{Theory of Computing Notes}
\author{Ollie Rees}

\begin{document}
	\maketitle
	\tableofcontents
	\section{Alphabet, Strings and Languages}
	\subsection{Alphabet}
	An alphabet is a set of symbols.
	\begin{eqnarray}
		\Sigma := \{0, 1\}
	\end{eqnarray}
	The above is the alphabet of binary strings. \newline
	An alphabet defines what characters a string can have.
	\subsection{Strings}
	Strings are a sequence of elements of the alphabet. 
	\begin{eqnarray}
		string := \{\sigma_0\sigma_1...\sigma_k\}. \sigma_i \in \Sigma
	\end{eqnarray}
	The set of all possible strings over an alphabet is defined below.
	\begin{eqnarray}
		\Sigma^{*} := \{s | \bigcap_{\sigma \in s} \in \Sigma\}
	\end{eqnarray}
	\subsection{Functions on Strings} 
	We can define a function that takes the number of letters in a string, where the letters are elements of an alphabet. 
	\begin{eqnarray}
		\# : \Sigma^* \rightarrow \mathbb{N} \\
		\#(s) = \#\{\sigma_0\sigma_1...\sigma_k\} = k
	\end{eqnarray}
	There is also concatenations of strings, defined as such.
	\begin{eqnarray}
		xy := {xy | x \in \Sigma^* . y \in \Sigma^*} 
	\end{eqnarray}
	\subsection{Languages} 
	A language is a subset of all the possible strings : $L \subseteq \Sigma^* $. \newline
	Two languages are equal when there is a bijective function between the two. \newline
	\section{Deterministic Finite Automata}
	\subsection{Definition}
	Deterministic Finite Automata (DFA) are machines with a finite number of states that accepts a set of strings over an alphabet. \newline
	The set of strings an automata depends on the construction of the automata, for example an automata may only accept states that end with 'a'. \newline
	The alphabet is represented, in an automata, by the path between states. \newline
	The mathematical definition of them is,
	\begin{eqnarray}
		M = \{Q, \Sigma, \delta, s, F \}
	\end{eqnarray}
	$ Q $ is a set where its elements are the states in the automata. \newline
	$\Sigma$ is the alphabet that the automata spans over, which is represented by the paths between states. \newline
	$\delta$ is the transition function, defined as $\delta: Q \times \Sigma \rightarrow Q $; $\delta(q_0, a) = q_1$ means that $q_0$ connects to $q_1$ via the letter 'a'. \newline
	$s$ is the starting state; in a DFA there is only one start state. \newline
	$F$ is the set of final states, also called accept states, since if a string ends on one it's accepted by the automata. \newline
	\subsubsection{Purpose}
	As briefly mentioned, the purpose of a DFA is to produce a language that follow a set of rules. \newline
	\subsubsection{Determinism}
	\subsection{Properties}
	\subsubsection{States}
	\subsubsection{Start State and Final States}
	\subsubsection{Transition Function}
	\subsubsection{Aggregate Transition Function}
	\subsection{Regular Languages}
	\subsubsection{Definition}
	\subsection{Operations on Automata}
	\subsubsection{Testing for Equality}
	\subsubsection{Complement}
	\subsubsection{Intersection}
	\subsubsection{Union}
	\subsubsection{Concatenation}
	\subsubsection{Asterate}
	\section{Nondeterministic Finite Automata}
	\subsection{Definition}
	\subsubsection{Literal definition}
	\subsubsection{Definition in relation to DFA}
	\subsection{DFA-NFA Relations}
	\subsubsection{Constructing DFA from NFA}
	\subsubsection{Consequences} 
\end{document}